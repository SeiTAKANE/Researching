\documentclass[a4paper,landscape]{jsarticle}
\special{papersize=\the\paperwidth,\the\paperheight}
\usepackage[dvipdfmx]{graphicx}
\usepackage{siunitx}
\usepackage{caption}
\usepackage {booktabs} 
\usepackage{pifont}
\usepackage{wasysym}
\usepackage{amssymb}
\usepackage{booktabs}
\usepackage{siunitx}
\newcolumntype{d}{S[input-symbols = ()]}

%start
\begin{document}
%記述統計
\begin{table}
\centering
\captionsetup{labelformat=empty,labelsep=none}
\caption{変数の記述統計}
\begin{tabular}{rlrrrr}
  \hline
 & 変数 & 平均 & 標準偏差 & 最小値 & 最大値 \\ 
  \hline
 & タクシーの営業収入(対2019年比) & 0.63 & 0.16 & 0.14 & 1.09 \\ 
 & 死亡率(人口10万人あたり) & 0.32 & 0.60 & 0.00 & 4.93 \\ 
 & 感染率 (人口10万人あたり) & 15.85 & 28.88 & 0.00 & 289.97 \\ 
 & 気温(月平均) & 14.84 & 8.08 & -4.40 & 30.70 \\ 
 & 失業率(月平均:\%) & 2.45 & 0.52 & 0.80 & 3.90 \\ 
 & 65歳以上人口(2019年) & 763531.92 & 680835.74 & 178000.00 & 3209000.00 \\ 
 & 人口密度(1haあたり, 2019年) & 6.57 & 11.95 & 1.00 & 63.00 \\ 
  & ICT従業者比率(2018年) & 0.01 & 0.01 & 0.00 & 0.06 \\ 
 \hline
\end{tabular}
\label{tb-ref}
\end{table}

%全体分析結果1
\begin{table}
\centering
\captionsetup{labelformat=empty,labelsep=none}
\caption{緊急事態宣言がタクシーの営業収入(対2019年同月比)に与えた影響}
\begin{tabular}[t]{lccccccc}
\toprule
  & モデル 1 & モデル 2 & モデル 3 & モデル 4 & モデル 5 & モデル 6 & モデル 7\\
\midrule
緊急事態宣言 & \num{-0.266}(\num{0.013}) & \num{-0.244}(\num{0.008}) & \num{-0.241}(\num{0.007}) & \num{-0.266}(\num{0.008}) & \num{-0.247}(\num{0.007}) & \num{-0.245}(\num{0.006}) & \num{-0.252}(\num{0.008})\\
LN(死亡率+1) &  &  & \num{-0.024}(\num{0.018}) & \num{-0.111}(\num{0.023}) &  & \num{-0.033}(\num{0.019}) & \num{-0.020}(\num{0.019})\\
LN(前月死亡率+1) &  &  & \num{0.009}(\num{0.014}) &  & \num{-0.008}(\num{0.017}) & \num{0.009}(\num{0.014}) & \num{0.046}(\num{0.019})\\
LN(感染率+1) &  & \num{0.000}(\num{0.003}) &  &  &  &  & \\
LN(前月感染率+1) &  & \num{-0.007}(\num{0.003}) &  &  &  &  & \\
気温 &  & \num{0.000}(\num{0.001}) & \num{0.000}(\num{0.001}) & \num{-0.005}(\num{0.000}) & \num{-0.001}(\num{0.001}) & \num{-0.001}(\num{0.000}) & \num{-0.002}(\num{0.001})\\
LN(失業率) &  & \num{0.056}(\num{0.033}) & \num{0.051}(\num{0.031}) & \num{-0.131}(\num{0.053}) & \num{0.016}(\num{0.038}) & \num{0.016}(\num{0.038}) & \num{0.044}(\num{0.036})\\
LN(65歳以上人口) &  & \num{0.027}(\num{0.015}) & \num{0.028}(\num{0.015}) &  &  &  & \\
LN(人口密度) &  & \num{-0.001}(\num{0.011}) & \num{-0.003}(\num{0.012}) &  &  &  & \\
LN(ICT従業者比率) &  & \num{-0.011}(\num{0.018}) & \num{-0.011}(\num{0.018}) &  &  &  & \\
定数項 & \num{0.679}(\num{0.005}) & \num{0.212}(\num{0.238}) & \num{0.196}(\num{0.240}) &  &  &  & \\
都道府県固定効果 & &  &  & \checkmark  & \checkmark & \checkmark & \checkmark \\
月固定効果 & &  &  &  & & &\\
年固定効果 & &  &  &  &  &  & \checkmark \\
\midrule
サンプルサイズ & \num{658} & \num{611} & \num{611} & \num{658} & \num{611} & \num{611} & \num{611}\\
R2 & \num{0.394} & \num{0.492} & \num{0.490} & \num{0.635} & \num{0.711} & \num{0.481} & \num{0.724}\\
自由度修正済決定係数(R2 Adj) & \num{0.393} & \num{0.485} & \num{0.483} & \num{0.605} & \num{0.685} & \num{0.476} & \num{0.698}\\
CR標準誤差 & & \checkmark   & \checkmark & \checkmark  & \checkmark & \checkmark  & \checkmark \\
\bottomrule
\end{tabular}
\end{table}

%比較分析結果
\begin{table}
\centering
\captionsetup{labelformat=empty,labelsep=none}
\caption{緊急事態宣言がタクシーの営業収入(対2019年同月比)に与えた影響:1回目と2回目の比較}
\begin{tabular}[t]{lccccc}
\toprule
緊急事態宣言  & 1回目(2020年4,5月) & & 2回目(2021年1,2,3月) \\
  & モデル1-1 & モデル 1-2 & モデル 2-1 & モデル2-2 & モデル 2-3\\
 \midrule
 緊急事態宣言 & \num{-0.295}(\num{0.007}) & \num{-0.297}(\num{0.007}) & \num{-0.021}(\num{0.027}) & \num{-0.011}(\num{0.022}) & \num{0.003}(\num{0.019})\\
LN(死亡率+1) & \num{-0.072}(\num{0.017}) & \num{-0.094}(\num{0.015}) & \num{-0.082}(\num{0.015}) & \num{-0.110}(\num{0.014}) & \num{-0.049}(\num{0.017})\\
LN(前月死亡率+1) & \num{0.001}(\num{0.038}) & \num{0.017}(\num{0.026}) & \num{-0.026}(\num{0.015}) & \num{-0.032}(\num{0.012}) & \num{-0.024}(\num{0.016})\\
気温 & \num{-0.002}(\num{0.001}) & \num{-0.003}(\num{0.000}) & \num{0.000}(\num{0.001}) & \num{-0.002}(\num{0.000}) & \num{-0.001}(\num{0.001})\\
LN(失業率) & \num{0.089}(\num{0.035}) & \num{0.105}(\num{0.018}) & \num{0.077}(\num{0.036}) & \num{0.041}(\num{0.032}) & \num{0.061}(\num{0.032})\\
LN(65歳以上人口) & \num{0.014}(\num{0.016}) &  & \num{0.021}(\num{0.018}) &  & \\
LN(人口密度) & \num{-0.009}(\num{0.013}) &  & \num{-0.003}(\num{0.013}) &  & \\
LN(ICT従業者比率) & \num{-0.010}(\num{0.019}) &  & \num{-0.010}(\num{0.018}) &  & \\
定数項 & \num{0.416}(\num{0.247}) &  & \num{0.288}(\num{0.268}) &  & \\
都道府県固定効果  &  & \checkmark &   & \checkmark & \checkmark\\
月固定効果  &  &  &  &  & \checkmark \\
\midrule
サンプルサイズ  & \num{470} & \num{470} & \num{517} & \num{517} & \num{517}\\
自由度修正済決定係数(R2 Adj)  & \num{0.630} & \num{0.830} & \num{0.114} & \num{0.526} & \num{0.737}\\
CR標準誤差 & \checkmark& \checkmark   & \checkmark & \checkmark  & \checkmark \\
\bottomrule
\end{tabular}
\end{table}

%マッチング結果
\begin{table}
\centering
\captionsetup{labelformat=empty,labelsep=none}
\caption{緊急事態宣言がタクシーの営業収入(対2019年同月比)に与えた影響:マッチング手法を用いた推定}
\begin{tabular}[t]{lccc}
\toprule
  & モデル1 & モデル2 & モデル3\\
\midrule
緊急事態宣言 & \num{-0.263}(\num{0.010}) & \num{-0.283}(\num{0.007}) & \num{-0.327}(\num{0.013})\\
LN(死亡率+1) & \num{-0.012}(\num{0.031}) & \num{0.080}(\num{0.023}) & \num{-0.115}(\num{0.051})\\
LN(前月死亡率+1) & \num{0.059}(\num{0.025}) & \num{0.056}(\num{0.020}) & \num{-0.157}(\num{0.055})\\
気温 & \num{-0.003}(\num{0.001}) & \num{-0.004}(\num{0.001}) & \num{0.004}(\num{0.002})\\
LN(失業率) & \num{-0.115}(\num{0.101}) & \num{0.017}(\num{0.049}) & \num{-0.023}(\num{0.090})\\
都道府県固定効果  & \checkmark  & \checkmark & \checkmark  \\
月固定効果  &  &  &   \\
マッチング手法  & MH最近傍マッチング(ATT) & MH最近傍マッチング(ATC) & CEM  \\
\midrule
サンプルサイズ & \num{238} & \num{238} & \num{132}\\
自由度修正済決定係数(R2 Adj) & \num{0.820} & \num{0.875} & \num{0.935}\\
CR標準誤差 & \checkmark& \checkmark   & \checkmark \\
\bottomrule
\end{tabular}
\end{table}
\newpage
\newpage
\newpage
\newpage

%t検定
\begin{table}
\centering
\captionsetup{labelformat=empty,labelsep=none}
\caption{主要場所への人の往来の比較}
\begin{tabular}{rlrrrr}
  \hline
 & 集計場所 & 1回目緊急事態宣言の平均(2020.4.6 - 5.25)( \% ) & 1回目緊急事態宣言以外の期間(-2021.3.31) ( \% ) & 差(95\%CI) & T値(絶対値) \\ 
  \hline
 & 小売店,娯楽施設 & \num{-31.64} & \num{-10.97} & ( \num{-23.0}, \num{-18.34} ) & \num{17.38}\\ 
 & 食料品,薬局 & \num{-1.94} & \num{-0.31} &  ( \num{-3.12}, \num{-0.15} )  & \num{2.16} \\ 
 & 乗換駅 & \num{-47.10}& \num{-23.69} &  (  \num{-26.03}, \num{-20.76} ) & \num{17.47} \\ 
 & 職場 & \num{-26.58} & \num{-13.01} &  ( \num{-17.30}, \num{-9.83} )  & \num{7.14} \\ 
 & 住宅 & \num{13.90} & \num{6.12} &  ( \num{6.30}, \num{9.10} ) & \num{11.03} \\ 
 & 公園 & \num{-4.52} & \num{-5.903} &  (  \num{-3.73}, \num{6.50} ) & \num{0.53} \\ 
 \hline
\end{tabular}
\label{tb-ref}
\end{table}

\end{document}





